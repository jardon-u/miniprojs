\documentclass[svgnames,smaller]{beamer}

\usepackage{criteo-theme}
\usepackage{tikz}

\usepackage[normalem]{ulem}
\usepackage{chronology}
\usepackage{fancybox}

\usepackage{biblatex}
\addbibresource{tutorial.bib}

\usepackage{listings}
\lstset{                         %
  language=C++,                  % choose the language of the code
  basicstyle=\ttfamily\small,     % the size of the fonts that are used for the
                                 % code
  numbers=left,                  % where to put the line-numbers
  numberstyle=\footnotesize,     % the size of the fonts that are used for the
                                 % line-numbers.
  stepnumber=1,                  % the step between two line-numbers.
  numbersep=5pt,                 % how far the line-numbers are from the code
  backgroundcolor=\color{gray!6}, % choose the background color. You must add
                                 % \usepackage{color}.
  showspaces=false,              % show spaces adding particular underscores
  showstringspaces=false,        % underline spaces within strings
  showtabs=false,                % show tabs within strings adding particular
                                 % underscores.
  frame=single,                  % adds a frame around the code
  tabsize=2,                     % sets default tabsize to 2 spaces
  breaklines=true,               % sets automatic line breaking
  breakatwhitespace=false,       % sets if automatic breaks should only happen
                                 % at whitespace.
  morekeywords={requires,in,constexpr,string},
  keywordstyle=\color{blue}\bfseries,
  commentstyle=\color{Gray},
  stringstyle=\color{red},
  escapeinside={\%*}{*)},         % if you want to add a comment within your code
  captionpos=top,
}
\renewcommand{\lstlistingname}{Code}

\newcommand*{\cpp}{\texttt{C++}}
\newcommand*{\csharp}{\texttt{C\#}}
\newcommand*{\jit}{\texttt{JIT}}

\newcommand{\msmall}{\fontsize{8}{10}\selectfont}

%% SLIDE INFORMATION------------------------------------------------------------

\title{\cpp 11 and Why You Care}
\author{Ugo Jardonnet}
\date{\today}


%-------------------------------------------------------------------------------
\begin{document}
%-------------------------------------------------------------------------------

\begin{frame}
  \titlepage
  %New features of \cpp and how they position in front of managed languages,
  %more precisely \csharp and Java
\end{frame}

\begin{frame}{Table of Contents}
  \tableofcontents
  % You might wish to add the option [pausesections]
\end{frame}

%-------------------------------------------------
\section*{Bootstrap}
%-------------------------------------------------

\begin{frame}[fragile]{Bootstrap}
  \begin{block}{The \cpp\ programming language:}
    \begin{center}
    "is a super-set of C with stronger abstraction" \cite{sutter-why}
    \end{center}
    \begin{chronology}[5]{1975}{2012}{3ex}{\textwidth}
      \event[1979]{\decimaldate{1}{10}{1982}}{1979: \texttt{C} with classes}
      \event{1983}{1983: The \cpp\ language}
      \event{1998}{1998: Standardization}
      \event{2003}{2003: International Standard}
      \event{2011}{2011: The new \cpp}
    \end{chronology}
  \end{block}
  % \csharp 2001, % Java 1995
  \begin{center}
    \shadowbox{The \cpp\ programming language follows the \textbf{zero-overhead principle} \cite{Stroustrup:1995:DEC:193198}}
  \end{center}
\end{frame}


%-------------------------------------------------
\section{Introduction}
%-------------------------------------------------

\begin{frame}[fragile]{Introduction}
  \begin{block}{What makes \cpp11 "feels like a new language",
      \citeauthor{stroustrup-think} \cite{stroustrup-think} ?}
    \begin{itemize}
    \item auto, lambda, range based for, ...
    \end{itemize}
  \end{block}

  \begin{alertblock}{Do we still need a language like \cpp ?}
    \begin{itemize}
    \item \csharp\  and Java focus on productivity
    \item \cpp\  is about performance and control
    \end{itemize}
  \end{alertblock}

  ... also the world is built on \texttt{C}/\cpp:
  \begin{itemize}
  \item \csharp, Java, Javascript, Ruby, Python, ... %FIXME: really Javascript ?
  \item Windows \sout{Core functions} \cite{sutter-why}
  \item Visual Studio %% FIXME: Visual Studio in C# ?
  \end{itemize}

    % http://www.ecma-international.org/publications/files/ECMA-ST/Ecma-334.pdf
    % Although C# applications are intended to be economical with regard to
    % memory and processing power
    % requirements, the language was not intended to compete directly on
    % performance and size with C or assembly language.
\end{frame}

%-------------------------------------------------
\section{Performance and Control}
%-------------------------------------------------

%-------------------------------------------------
\subsection{Lambda: Control Closure Capture}

\begin{frame}[fragile]{\cpp\ Lambda}
  \begin{block}{Definition}
    [capture](parameters) \texttt{->} return-type \{body\}
  \end{block}
  \begin{block}{Example \footnotemark}
  \begin{lstlisting}
  std::sort(begin(v), end(v), [](const T& a, const T& b) {
      return a.x < b.x;
  });
  \end{lstlisting}
  \end{block}
  \footnotetext{[](auto e) is not valid in \cpp\ nor any king of variadic
    lambda}
  %% FIXME: Why not [](auto e) why does it works in \csharp ? Java ? Scala ?
\end{frame}

\begin{frame}[fragile]{\cpp\ Lambda : Control Closure Capture}
  A closure is a function capturing free variables in the lexical environment
  \begin{lstlisting}[title={Lambda Closure}, basicstyle=\ttfamily\msmall]
 []      // The use any external variables is an error.
 [x, &y] // x is captured by value, y is captured by reference
 [&]     // Implicitly capture by reference
 [=]     // Implicitly capture by value
 [&, x]  // x is explicitly captured by value. Other variables will be captured by reference
 [=, &z] // z is explicitly captured by reference. Other variables will be captured by value
  \end{lstlisting}
  \begin{alertblock}{Question:}
    In \csharp, how do lambdas capture variables of their closure ? Python ? Java ?
  \end{alertblock}
  %% C# by reference, Java local and anonymous classes by value (must be final)
  %% Caml by value, python by reference, php you can choose...
\end{frame}

%-------------------------------------------------
\subsection{Control Compile-Time vs Run-Time execution}

\begin{frame}[fragile]{Control Compile-Time vs Run-Time execution}
  \begin{block}{\textbf{Compile-Time Execution}}
    \begin{lstlisting}[title={constexpr \footnotemark}]
  constexpr factorial(int n) {
    return n > 0 ? n * factorial( n - 1 ) : 1;
  }
    \end{lstlisting}
    \begin{lstlisting}[title={static\_assert}]
  template <typename T>
  struct Check {
    static_assert(sizeof(int) <= sizeof(T),
                  "T is not big enough!");
  };
    \end{lstlisting}
  \end{block}

  \footnotetext{\texttt{factorial} may also be executed at run-time depending on \texttt{n}}
\end{frame}

\begin{frame}[fragile]{Control Compile-Time vs Run-Time execution}
  \begin{block}{\textbf{Compile-Time Metaprogramming \footnotemark}}
  \begin{lstlisting}[title={decltype}]
  auto mult(T1 a, T2 b) -> decltype(a * b) {/*...*/};
  \end{lstlisting}
  \begin{lstlisting}[title={\texttt{<type-trait>}}]
  template <typename T>
  typename enable_if<is_integral<T>::value, void>::type
  f(T) { cout << "integral" << endl; }

  template <typename T>
  typename enable_if<!is_integral<T>::value, void>::type
  f(T) { cout << "not integral" << endl; }
  \end{lstlisting}
  \end{block}
  \footnotetext{No run-time metaprogramming in \cpp\ }
\end{frame}


\begin{frame}[fragile]{Concepts}
  \begin{alertblock}{Should have been like this}
  \begin{lstlisting}
template <Integral T> f(T) { /*...*/ }
  \end{lstlisting}
  or this
  \begin{lstlisting}
template <typename T> requires Integral<T> f(T) { /*...*/ }
  \end{lstlisting}
  \end{alertblock}

  but are more like this
  \begin{lstlisting}
template <typename T>
typename enable_if<is_integral<T>::value, void>::type
f(T) { /* ... */ }
  \end{lstlisting}
though can be more something like this {\cite{martniho-enable}}
  \begin{lstlisting}
template <typename T, EnableIf<is_integral<T>>...> f();
  \end{lstlisting}
\end{frame}

\subsection{Control Memory Management}

\begin{frame}[fragile]{Control Memory Management}
  \begin{block}{Fine grain memory management through Smart Pointers \footnotemark}
  \begin{lstlisting}[title={\texttt{unique\_ptr}}]
  \end{lstlisting}

  \begin{lstlisting}[title={\texttt{shared\_ptr}}]
  \end{lstlisting}

  \begin{lstlisting}[title={\texttt{weak\_ptr}}]
  \end{lstlisting}
  \end{block}

  \footnotetext{Java Dispose, \csharp\ using, D scope, ...}
  % interestingly D supports both Garbage collection and RAII
\end{frame}

%-------------------------------------------------
\section{Productivity Oriented Languages}

\begin{frame}[fragile]{Productivity Oriented Languages}
  %%When it comes to performance optimization, C++ always chooses “prevention,”
  %%and managed languages choose “cure”, Sutter
  \begin{block}{Productivity} \msmall
    \begin{itemize}
      \item Virtual Machine
      \item Garbage Collection
        %% Some algorithms that are faster with GC
        %% ABA problems for high-performance lock-free coding
      \item MetaData
      \item linq
    \end{itemize}
  \end{block}
  \begin{block}{Control}
    \begin{itemize}
      \item Cache locality \footnotemark[4] and Memory Alignment ?
      \item Optimization completely relies on \jit\ compiler \footnotemark[5]
        \begin{lstlisting}[title={\small
            \texttt{jdk7/mytl/jdk/src/share/classes/sun/tools/javac/Main.java}
            \footnotemark[6]}, basicstyle=\ttfamily\msmall]
if (argv[i].equals("-O")) {
  // -O is accepted for backward compatibility, but
  // is no longer effective. [...]
}
        \end{lstlisting}
    \end{itemize}
  \end{block}
\footnotetext[4]{L1 Cache ref is 0.5ns, Main Memory ref 100ns, disk seek 10,000,000ns
\cite{slatkin-numbers}}
\footnotetext[5]{\jit\ ``which is in the business of being fast''
  \cite{sutter-jit}.}
\footnotetext[6]{See also {\footnotesize \citetitle{tsunasblog-dev}
    \cite{tsunasblog-dev}}}
\end{frame}

\begin{frame}[fragile]{Productivity Oriented Languages}
  \begin{block}{Performance}
    \begin{itemize}
    \item \jit\ compilers are getting better.
    \item \cpp\ compilers are getting better.
    \item NextGen/AOT compilers are coming.
    \item ... in fact the trend is \textit{Going Native} \cite{Ms-going-native}.
    \end{itemize}
  \end{block}

  \begin{alertblock}{By the way ...}
    \begin{itemize}
    \item Linq is a nice tool but also a performance killer \footnotemark .
    \item Access to MetaData is even slower than people say (\texttt{Enum.ToString?}).
    \end{itemize}
  \end{alertblock}
  \footnotetext{Enumerable.Cast: Top 5 most consuming function of Web.Widget}
\end{frame}

%-------------------------------------------------
\section{New  language features and librabries}
%-------------------------------------------------

%-------------------------------------------------
\subsection{\cpp11 feels like ...}

\begin{frame}[fragile]{... in fact \textbf{\texttt{C++11}} feels like Python}
  %% DIXME: What does cross do ?
  \begin{lstlisting}[title={python3}]
  v = [1,2,3,4,5]
  for e in v: print("- ", e)

  unitlist = [cross(rs, cs) for rs in ['ABC','DEF','GHI'] \ 
                            for cs in ['123','456']]

  \end{lstlisting}
  \begin{lstlisting}[title={\cpp11}]
  auto v = {1,2,3,4,5};                 // initializer list
  for (auto e: v) println("- ", e);

  vector<vector<string>> unitlist;
  for (auto rs: {"ABC"_vec,"DEF"_vec,"GHI"_vec}) {
    for (auto cs: {"123"_vec, "456"_vec}) {
      unitlist.push_back(cross(rs, cs));
    }
  }
  \end{lstlisting}
\end{frame}

\subsection{User-defined Literal}

\begin{frame}[fragile]{User-defined Literal}
  \begin{lstlisting}[title={\texttt{""\_vec}}]
  constexpr vector<string>
  operator"" _vec(char const* str, size_t N)
  {
    vector<string> output;
    for_each(str, str+N, [=, &output] (char c) {
      output.push_back(string(1, c));
    });
    return output;
  }
  \end{lstlisting}

  \begin{block}{Good or Bad (?), finally we've got \texttt{std::string} literal !!}
    \begin{lstlisting}
  std::string operator"" _s(const char * str, size_t len)	{
	    return std::string{str, len};
  }
    \end{lstlisting}
  \end{block}
\end{frame}


%-------------------------------------------------
\subsection{Variadic templates}

\begin{frame}[fragile]{Variadic templates}
  Variadic templates are a type-safe\footnotemark[1] alternative to vargars and
  much more. %FIXME: bof

  \begin{lstlisting}[title={println}]
  void println() {
    std::cout << std::endl;
  }
  template <typename T, typename ...Args>
  void println(const T& value, const Args&... args) {
    std::cout << value << " ";
    println(args...);
  }
  \end{lstlisting}

  A variadic template is a type queue.
  Remember \cpp\ metaprogramming is functional style \footnotemark[2].
  \footnotetext[1]{This mean I can write a specialized version for a given type T}
  \footnotetext[2]{D metaprogramming is imperative style}
\end{frame}

%-------------------------------------------------
\subsection{Threading}

\begin{frame}[fragile]{Threading: Future-Based Model}
  New memory model \cite{mem-model} $=$ stronger guarantees $=$ more aggressive optimization
  \begin{lstlisting}[title={Taste of \cpp11 \cite{sutter-modern}}]
  string flip(string s) {
    reverse(s.begin(), s.end());
  }

  int main() {
    vector<future<string>> v;
    v.push_back(async([] { return flip(   " ,olleH"); }));
    v.push_back(async([] { return flip(" egdelwonK"); }));
    v.push_back(async([] { return flip("\n!rebmahc"); }));

    for (auto& e : v) {
      cout << e.get();
  }
  \end{lstlisting}
%%  \begin{quotation}
%%    Asynchronous programming is what the doctor usually orders for unresponsive
%%    client apps and for services with thread-scaling issues. \  This usually means a
%%    bleak departure from the imperative programming constructs we know and love
%%    into a spaghetti hell of callbacks and signups. \csharp\ and VB????? are putting
%%    an end to that, reinstating all your tried-and-true control structures on top
%%    of a future-based model of asynchrony.
%% 
%%    Mads Torgersen, Language PM for \csharp\ at Microsoft
%%  \end{quotation}
\end{frame}

\section*{Bibliography}
\begin{frame}[allowframebreaks]{Bibliography}
\printbibliography
\end{frame}

\end{document}
